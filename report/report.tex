\documentclass[12pt]{article}

\usepackage{graphicx}
\usepackage{url}
\usepackage{amsmath}
\usepackage[cm]{fullpage}
\usepackage{calc}
\usepackage{subfig}

\title{\textbf{Wireless and Mobile Computing, Assignment 5}}
\author{Thomas Richner, 09-920-539}
\begin{document}
\maketitle

\section{Sound Analysis}
 
\subsection{Time domain}
From the amplitude analysis (see figure \ref{fig:graphs}) in the time domain we can see that song 2 has a higher amplitude on average and a more constant amplitude than song 1.
\subsection{Frequency Domain}
Song 2 has a relatively high spectral power density of -70dB at lower frequencies i.e. below 5 kHz, but above the power density is mostly below -100 dB. \

Song 1 has a more balanced power density which is similar at around -90dB from 5 kHz to 20 kHz.

\subsubsection{Conclusion}
In order to hide information there must be a signal to hide in. Song 1 has only short bursts of power, which makes transmission of an entire packet relatively unlikely, Song 2 on the other hand has a more or less constant signal with acceptable amplitude below 4 kHz.

Hence I would propose to use Song 2 as the signal to hide in. Furthermore I would choose a frequency between 1 kHz and 4 kHz.


\begin{figure}
\centering
\subfloat[frequency domain 1]{
  \includegraphics[width=80mm]{img/1_freq_domain.png}
}
\subfloat[frequency domain 2]{
  \includegraphics[width=80mm]{img/2_freq_domain.png}
}
\hspace{0mm}
\subfloat[time domain 1]{
  \includegraphics[width=80mm]{img/1_time_domain.png}
}
\subfloat[time domain 2]{
  \includegraphics[width=80mm]{img/2_time_domain.png}
}
\hspace{0mm}
\subfloat[spectrogram 1]{
  \includegraphics[width=80mm]{img/1_spectrogram.png}
}
\subfloat[spectrogram 2]{
  \includegraphics[width=80mm]{img/2_spectrogram.png}
}
\caption{sound analysis \label{fig:graphs}}
\end{figure}

\section{Song Quality}
I can not distinguish the modified song from the original with a non-negligible higher probablility than 50\%. This might be due to the already bad quality of my sound equipment.

\section{Received Signal}
\subsection{Song 1}
Decoding results:
\begin{verbatim}
block offset: 3298
Sync offset: 1
Max sync coefficient: 0.96627
Elapsed time decoding: 26.3004 seconds.
 
Bit Error Rate (BER): 0%
\end{verbatim}
Song 1 can be fully decoded without any bit-errors.

\subsection{Song 2}
Decoding results:
\begin{verbatim}
block offset: 3309
Sync offset: 1
Max sync coefficient: 0.71403
Elapsed time decoding: 42.4579 seconds.
 
Bit Error Rate (BER): 0.3%
\end{verbatim}
Song 2 has some bit-errors, mostly due to the fact that the first second of the song has no signal to hide in.


\section{Feedback}

\begin{description}
  \item[Difficulty] \hfill \\ veery easy
  \item[Clarity] \hfill \\ Very clear
  \item[Time Spent] \hfill \\  2h
  \item[Conclusion] \hfill \\ Fun assignment, instructions how to play and record it would have been nice.
\end{description}

\end{document}
